\section{Node Types}
Currently, in Ethereum there are mainly three types of Nodes: full nodes,
archive nodes and light nodes.

\paragraph{Full nodes}
The full nodes are the nodes we describe throughout our work. They store the
whole blockchain (comprising of block headers and bodies) and have full
copy of the most recent state and verify and process every transactions. 
Eventually a full node can be a miner, if it try to find new blocks.

\paragraph{Archive nodes}
Archive nodes are full nodes that store also the state tree for each block. It
is used by block explorers, e.g.
etherscan\footnote{\url{https://etherscan.io/}}.

\paragraph{Light nodes}
Light nodes store only the headers of the blocks of the blockchain. They do
neither store the blocks' bodies nor the state. The idea behind light node,
is that they use the other peers in the network as a distributed hash table 
(DHT). In essence, they know the hash of the information and request the
desired information from the peers they know with an on-demand logic. 

For example, a light node can retrieve the state of an account at a 
desired block, by recursively requesting the content of the world state, 
starting from its hash, that is contained in the block's header
(\autoref{fig:world-state}). We refer to~\cite{bib:light-client}, to read 
details about what type of other information a light node can retrieve and how.

This type of client assume that there are full-nodes in the network that 
supports also the Light Ethereum Subprotocol (LES)~\cite{bib:les-protocol}, 
that have message types to retrieve data on-demand.