\subsection{Ethereum Virtual Machine}
The Ethereum Virtual Machine (\textbf{EVM}) is an abstract computing machine
that enables the nodes of the Ethereum network to execute smart-contract
codes.
The EVM's specification and the description of all the byte-code instructions
are contained in the yellow paper~\cite{wood2018ethereum}.
Like the JVM, the EVM is a stack-based machine.
The word size and the stack item size are $256$--bit\footnote{The motivation
for this choice is to facilitate the Keccak-256 hash scheme and elliptic
curve computation that are pervasive in Ethereum.},
and the stack capacity is $1024$.
The EVM memory consists of a simple byte array and exists only during the
execution. The size of this array is allocated using an on-demand logic.
The storage is a key value hash table that is persistent and is
part of the state of the contract account. The program is stored on the
blockchain and can therefore not be modified\footnote{There exist
techniques to provide new version updates as explained
here:
\url{https://ethereum.stackexchange.com/questions/2404/upgradeable-smart-contracts}}.
In addition to the aforementioned structures the EVM has at its
disposal an input field, which is provided by the initiator of the execution
(e.g. the transaction that starts the execution). The EVM is depicted
in Figure \ref{fig:evm}.

Unlike the bitcoin \textit{Script} language~\cite{bib:masteringbitcoin} the
Ethereum Virtual Machine can express and supports the execution of loops.
To avoid the abuse of the resources (CPU and storage) of the full nodes
forming the network, Ethereum introduces the concept of
\textbf{gas}~\cite{wood2018ethereum}. In this
execution model each instruction and increase in the size of the memory or
storage are bound to a cost expressed in gas. The price of a gas unit
is known as \textbf{gas price} and is bound to a particular execution.
Indeed this price is specified in the transaction or in the message call
which originates the execution. The higher this price the higher the
possibility that the  transaction will be included in the blockchain.
Usually the miners advertise the minimum gas price they are willing to accept.
Another important concept specified in the transaction is the
\textbf{gas limit}, i.e. the maximum gas amount the executor is willing to
consume for this particular execution.

Thanks to this brief introduction we can already understand some crucial
mechanisms involved in Ethereum and their motivation:
\begin{itemize}
	\item A transaction is not valid if the originator of a transaction have
	a balance lower than \verb|gas_price * gas_limit| \verb|+|
	\verb|transaction_value|:
	it means that it cannot afford the execution. In fact this value is
	deducted upfront and later the remaining gas are refund.
	\item If during the execution the gas limit is exceeded an Exception
	(\textbf{Out of gas exception}) is raised, and the funds are not returned,
	because the execution already took place.
\end{itemize}




