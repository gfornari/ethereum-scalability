\subsubsection{Smart contract}
\label{sec:smart-contract}

The smart contracts in Ethereum are part of the application layer. They can be
referred to as implementing the business logic of the applications that runs on
the Ethereum system. A smart contract implements functions and has a state which
can be persisted and accessed in a subsequent execution. They can implement
\emph{almost} any function which can be implemented in any Turing-complete
machine. The difference is due to the gas needed for a transaction to be
executed: since the gas it can not be infinite, there will be always an upper
bound on the possible total amount of computation~\cite{wood2018ethereum}.

Usually, a smart contract is written in a high-level language (e.g.
Solidity\footnote{\url{https://github.com/ethereum/solidity}} or
Vyper\footnote{\url{https://github.com/ethereum/vyper}}), then compiled in its
bytecode. The contracts communicate with each other and with external actors
through their Application Binary Interface (ABI).

% TODO specify better these ABI. Maybe add a nice diagram about two contracts in
% their evm communicating through the ABI (with an external actor as well)
