\subsection{Tests}

To measure the scalability of Ethereum we prepared some tests to study
the maximal throughput and the size of the blockchain with different
configurations.
To study the scalability of one permission-less blockchain system such as
Ethereum one should prepare some tests using thousand of nodes~\cite{}.
Since we do not dispose of so many resources we take inspiration from
Blockbench~\cite{blockbench}, which compares the performance and scalability
of Hyperledger~\cite{} and Ethereum in a \emph{private} scenario, that is when 
we take into consideration a limited number of authenticated nodes.

We tried to use the public available Blockbench 
repository\footnote{\url{https://github.com/ooibc88/blockbench}}
but we did not manage to configure it, because of a lot of hard-coded
configuration variables and the lack of a well-written documentation.
Therefore we desisted and wrote our own system.



\paragraph{Genesis file}

The genesis file contains useful information to create (deterministically)
the genesis block. It contains several parameters, such as the maximum
gas limit for the files, the difficulty and an initial allocation of Ether
for some accounts\footnote{This possibility has been used for the so-called
ICOs (Initial Coin Offering) used to obtain fiat currency to finance
the project.}. Each node of the network should be initialized with
the same genesis file. The main network and the official Ethereum test
networks use hard-coded values.

\subparagraph{Difficulty}
As already described in \autoref{}, the difficulty is an adaptive parameter 
that determines how much effort should be invested in the creation of a new
block.
In our test we use the same hardware and same operating system in all nodes
and for all miners.
Therefore to find a suitable start value for the difficulty we ran five times 
a single machine in mining mode for 30 minutes and took the average difficulty
of the last mined block as the value for the test.


\subsubsection{Test Configuration}


