\subsection{Tests}

To measure the scalability of Ethereum we prepared some tests to study
the maximal throughput and the size of the blockchain with different
configurations.
To study the scalability of one permission-less blockchain system such as
Ethereum one should prepare some tests using thousand of nodes~\cite{}.
Since we do not dispose of so many resources we take inspiration from
Blockbench~\cite{blockbench}, which compares the performance and scalability
of Hyperledger~\cite{} and Ethereum in a \emph{private} scenario, that is when 
we take into consideration a limited number of authenticated nodes.

We tried to use the public available Blockbench 
repository\footnote{\url{https://github.com/ooibc88/blockbench}}
but we did not manage to configure it, because of a lot of hard-coded
configuration variables and the lack of a well-written documentation.
Therefore we desisted and wrote our own system.


\subsubsection{Test Configuration}

To keep the configuration easy we opted for a classic master-slave logic.
The master, i.e. the initiator of the test, uses the ssh protocol
to run commands on the remote machines.
Similarly to~\cite{blockbench} we distinguish the roles \emph{miner} and
\emph{client}. The former is accountable to generate new blocks while the
latter creates and propagates transactions and both roles verify the 
blocks\footnote{To reduce the number of test variables we consider only 
	full nodes}.
We can assign multiple roles to a machine. In this case we run
\emph{one geth instance} for each role.



In each test we distinguish at least two phases:
\begin{enumerate*}
	\item the setup, and
	\item the test itself.
\end{enumerate*}
The former is used to initialize the local machine with the right genesis
file and the data structures to mine and validate the blocks.

\paragraph{Genesis file}

The genesis file contains useful information to create (deterministically)
the genesis block. It contains several parameters, such as the maximum
gas limit for the files, the difficulty and an initial allocation of Ether
for some accounts\footnote{This possibility has been used for the so-called
ICOs (Initial Coin Offering) used by the Ethereum Foundation to obtain fiat
currency to finance the project.}. Each node of the network should be 
initialized with the same genesis file. 
The main network and the official Ethereum test
networks use hard-coded values.


\subparagraph{Difficulty}
As already described in \autoref{}, the difficulty is an adaptive parameter 
that determines how much effort should be invested in the creation of a new
block.
In our test we use the same hardware and same operating system in all nodes
and for all miners (we deliberately used only one thread for mining).
Therefore to find a suitable start value for the difficulty we ran a single
machine in mining mode 


\subparagraph{Timestamp}
The timestamp of the genesis file, that corresponds to the one of the genesis,
block influences the difficulty of the first blocks.
Since, for time constraints, we want to run each test for few minutes and
at the same time keep the difficulty found with the initial tests,
during the setup phase the initiator reads its timestamp and gives it
as parameter to the peers.


