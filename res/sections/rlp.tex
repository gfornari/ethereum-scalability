\subsubsection{Serialization Algorithm}
\label{sec:marshaling}
The Recursive Length Prefix (RLP) encoding algorithm is a fundamental building
block in the Ethereum system. It is used both to serialize the content of the
UDP and TCP packets sent, as described in~\autoref{sec:network-layer}
and~\autoref{sec:rlpx-transport-protocol}, and to reach a bit level consensus on
the World State through the blockchain, as described
in~\autoref{sec:world-state}.


This algorithm is used \emph{only} to encode byte arrays of arbitrary length.
It does neither try to deal with types nor considering signed integers and
floating numbers~\cite{wood2018ethereum}. The interpretation of the values
is completely dependent on the message in the protocol, which should
also specify the byte-size of the structures involved.
According to the documentation~\cite{bib:design-rationale} RLP has been chosen
for its simplicity and its byte level consistency.
For the formal specification of the algorithm we refer
to the Yellow Paper~\cite[Appendix B]{wood2018ethereum} and the RLP
documentation~\cite{bib:ethereumrlpspec}.
