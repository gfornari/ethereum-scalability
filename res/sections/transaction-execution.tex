\subsection{Transaction Execution}
The transaction execution is the mechanism through which the world state is
updated. It represents a transition from one
valid state to another valid state. Before executing a transaction it
has to pass some simple validity checks, e.g.\ the transaction should be a
well-formed RLP encoded string and should have a balance big enough to afford
the transaction.
Moreover, since a transaction should be included in a block and the block
has in turn its own gas limit, it should be also true that the sum 
of the accumulated gas used by the already included transactions and the gas
limit of this transaction are smaller than the block's gas limit.

Before performing any operation the nonce of the 
account that starts the execution is incremented by one and its balance is
reduced by the product of the gas limit and the gas price. This modification
is irreversible.
After this first state modification we should  make a distinction between
contract creations and message calls, depending on whether the receiver is
empty or not.
During the execution of the transaction the executor keeps track of the
\textbf{transaction substate}, i.e.\ some important information that
are later used to complete the transaction execution.
The transaction substate includes the \textbf{touched accounts}, the set of
accounts that will be discarded following the completion (\textbf{self-destruct
set}) and more notably the
\textbf{refund balance}, that corresponds
to the difference between the allocated gas amount and the effectively used
gas augmented with the gas returned for removing elements from the world
state. After the completion of the transaction this amount is returned to the
initiator of the transaction at the transaction's gas price.
The gas used is given to the beneficiary address (i.e. the miner), who builds 
and finalize the block.

To reach the final state it is necessary to delete the self destruct state and
the touch state that becomes empty or dead.
