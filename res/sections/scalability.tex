\section{Scalability}
\label{sec:scalability}
The aim of this section is to study the scalability of Ethereum.
Although the scalability of blockchain, and of Ethereum in particular, is a
well-known problem and a major concern in the respective communities, in the
literature there are a limited number of \emph{scientific papers} that
addresses this problem.
We give a little survey in \autoref{sec:background}.

In the remainder of the Section we will analyze the scalability of Ethereum by
keeping into consideration the 3 axes of the cube of the
scalability~\cite{bib:art-of-scalability}.


\subsection{Background}

In this section firstly, we point out the usual concerns about blockchain
technology and in particular Bitcoin and finally we analyze the related works
regarding Ethereum.

\paragraph{Bitcoin} 
One major concern about Bitcoin is the limited \textbf{transaction
throughput}~\cite{bib:blockchain-challenges-opportunites-survey, bib:taxonomy}.
Therefore only the more profitable transactions are included in the distributed 
ledger by the
miners~\cite{bib:blockchain-challenges-opportunites-survey,wood2018ethereum}.
This limit is due on the one hand to the $1$ MB \textbf{block
	size}~\cite{bib:masteringbitcoin} and on the other
hand to the \textbf{block interval} that is about $10$ minutes.
Unfortunately, increasing the former increase the propagation time and
the possibility of new branches and decreasing the latter augment the
bandwidth usage and storage needed.
Indeed the \textbf{storage} space required by the blockchain is continuously
increasing.

Finally, a number of confirmation blocks are
needed to be sure that the transactions are really 
confirmed~\cite{bib:taxonomy}, i.e. it is unlikely that the block in which
they are included can be discarded.
In Bitcoin the 
confirmation number is 6 blocks that corresponds roughly to an hour~\cite{}.
 
\paragraph{Ethereum} Vukoli\'c~\cite{bib:the-quest} states that the introduction of
smart-contracts 
in Ethereum make the problem very similar to database replication and in 
particular \emph{state machine replication}. Also Wood~\cite{wood2018ethereum}
argue that it is very difficult to reach a high degree of scalability by
parallelizing transactions in this system because it is essentially a state
transaction machine.
Indeed the state in Ethereum influence the smart contract execution and
therefore the majority of transactions are dependent from previous ones, thus
making Ethereum \textbf{stateful}.


%*** et al. \cite{bib:sok:ethereum} individuate that scalability of blockchain
%is concerned with two major objectives: on the one hand increasing the
%transaction throughput and on the other hand
%decreasing the storage and bandwidth usage.

