\subsubsection{Contract Creation}
The contract creation is the act with which a new contract is deployed
on the system. It can be triggered either by a transaction or during the 
execution of an existing contract. In the remainder of this section we refer
to \textit{sender} as the entity who creates the account, that does not 
necessarily correspond to the author of the transaction.

Firstly, the 160-bit identifier for the contract account is determined as
function of the sender and its nonce. The value specified in the contract
creation is transferred from the sender to the brand new contract account.
Afterward the \textbf{init} code is executed on the Ethereum Virtual Machine.
This  piece of code initiate the storage of the contract account and returns the
\textbf{body}, i.e. the contract code that will persist on the account state.
If during the execution an out-of-gas exception occurred the state is reverted
to the initial state, as if the contract creation did not take place.
If the execution of the init code is successful the creator must pay an amount
of gas proportional to the size of the \textbf{body}, because it must be
stored by all the full nodes. If the gas remained after the execution are not
enough to afford this cost or the body of the contract code is too big the
state is reverted~\cite{wood2018ethereum}.
If all the whole procedure succeed the hash of the body is saved on the
contract account state. Clearly the full nodes use this hash as a key for the 
contract code, as showed in \autoref{fig:world-state}.
 