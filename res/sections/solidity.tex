\section{Solidity}
Solidity~\cite{bib:solidity-docs} is the most popular
high-level language used in the Ethereum ecosystem.
It is a high-level statically typed language that is compiled to EVM bytecode.
It takes inspiration from various languages, such as JavaScript, Python and C++.

In addition to the usual mathematical and logical operations Solidity has
several peculiar variables and functions that resides on the global scope and
allow the programs to access to the data shown in \autoref{fig:evm}. 
One of this specially crafted global variables is the \texttt{msg} variable,
that allows the programs to access the message call data,
such as the code invoker's address (\texttt{msg.sender}) and the amount of 
money sent along with the invocation (\texttt{msg.value}).

In order to provide an intuitive insight about Solidity we provide a little
example in \autoref{listing:example}, that shows some key features of solidity.

\begin{figure}
\lstinputlisting[language=Solidity, caption={Solidity
	Example}, label=listing:example]{./res/code/example.sol}
\end{figure}

As it is easy to see the contracts are the building block of the language and
resemble objects. Indeed in Solidity one contract can also inherit from other
contracts.
The fields of the contract are stored in the contract's storage. In the example
there are two persistent variables \verb|x| and \verb|owner|. While the former
is marked as public the latter is marked as private. Thus, the compiler
creates a getter \verb|x()| for variable \verb|x| but not for variable 
\verb|owner|. It should be noticed that the \verb|owner| variable can still
be read from outside the EVM\footnote{It is part of the contract's storage,
which in turn is part of the world state that is replicated in each node.}, 
but it cannot be read from other contracts that run on the EVM.


After the definition of the field variables we can see the \textbf{constructor}
method. It is run only once upon creation and returns the code of the contract
as we described in~\autoref{sec:contract-creation}. If it is not defined
an empty constructor is assumed.

The \textbf{events} are a high-level interface for the EVM's logging 
facilities. Events are emitted by smart contracts to notify external 
applications, that can listen to them to perform some application-specific
operations. Logs are visible from the outside because they are stored
in the blockchain as part of the transaction receipt which emitted them.

After the definition of the events, a the \textbf{modifier} \verb|onlyOwner| is
defined. A modifier is a piece of code that is executed before or after a 
function decorated with it is called. Modifiers can be used to implement final
state machines~\cite{bib:solidity-docs} and conditions that should be fulfilled 
to execute or after execution of a particular category of functions, e.g. 
pre-conditions and or post-conditions on state.

The \textbf{functions} are an abstraction of the language. They can be invoked 
either  \texttt{internally}, that is inside the same call stack activation 
record with a JUMP instruction or \texttt{externally}, that is by creating a
new message  call. In the former case the machine state is the same of the
invoker, while in the latter the code is executed in a new machine state, that is by creating a new activation record on the call stack, as discussed in
\autoref{sec:evm}.
The first 4 bytes of the keccak256 hash of the signature of external function
(name and parameter types) is used as unique identifier (\textbf{selector}) of
a function. When an actor wants to invoke a function externally he should 
provide its selector in the message data and follow the ABI
format~\cite{bib:solidity-docs}.
Essentially the Solidity compiler puts at the start of the compiled code a
switch-statement that checks if the first four bytes of the message call data
match a known (external) function. If it is the case the program jumps to the
definition of the matched function otherwise a so-called fallback function is 
invoked.
The programmer may specify the fallback function by writing an unnamed function
that does not take parameters. Since the signature depends on the number and
the type of arguments, function overloading is easily implemented by using
the new obtained selector.

Apart from the aforementioned visibilities (external and internal)
we distinguish also between public and private functions. The former can be
called either externally or internally, while the latter can be called only 
internally and are not visible to contracts that are below in the inheritance
hierarchy.

Apart from visibility the function header may contain a state mutability flag:
we distinguish between \texttt{pure}, \texttt{view} and \texttt{payable}. 
The first indicates that the function will not access the state, the second
promises to access the state in a read-only fashion while the latter indicates 
that the function is capable of receiving money.
Pure and view functions can be invoked also externally, i.e. without executing
the EVM and spending gas, indeed in that case they are either returning a
constant or are reading the storage, but if they are invoked during the
execution of a transaction it spends gas.




