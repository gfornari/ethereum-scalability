\subsection{To scale or not to scale}
% discutere se la scalabilità in Ethereum è desiderabile oppure c'è scritto
% nello statuto che non importa a nessuno per gli obiettivi del sistema e per
% cosa è pensato. Inoltre, suggerire quali porte potrebbe aprire assumendo di
% aggiungerla mantenendo le proprietà attuali del sistema.

Before start discussing about the scalability in Ethereum, it is worth clarify
if it is a real concern or not.

In \cite{wood2018ethereum}, Gavin Wood defines Ethereum as ``a project which
attempts to build the generalised technology; technology on which all
transaction-based state machine concepts may be built.'', suggesting that, even
systems for which is expected a high number of transactions may be built with
Ethereum. Following this aim, one typical example is the case of Visa. Visa is
capable of handling 56000 transactions per
second\footnote{https://usa.visa.com/dam/VCOM/download/corporate/media/visa-fact-sheet-Jun2015.pdf},
while Ethereum can roughly process 15 transactions per
second\footnote{https://medium.com/l4-media/making-sense-of-ethereums-layer-2-scaling-solutions-state-channels-plasma-and-truebit-22cb40dcc2f4}.
This comparison points out that Ethereum is far away from possibly implementing
all such systems, showing that the scalability from this perspective seems
something to care about.

In January 2, 2018, the Ethereum Foundation announced two subsidy programs both
intended to fund projects on the scalability research and
development\footnote{https://blog.ethereum.org/2018/01/02/ethereum-scalability-research-development-subsidy-programs/}
recognizing that ``scalability as perhaps the single most important key
technical challenge that needs to be solved in order for blockchain applications
to reach mass adoption''. This shows how all the Ethereum community founders
included, believes that the scalability is a major concern and an actual
bottleneck in the employment of the technology.
