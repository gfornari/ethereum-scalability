\section{Node types}
\label{sec:node-types}
Currently, in Ethereum there are mainly three types of nodes: full nodes,
archive nodes and light nodes. Hereafter, we describe each type of node and
explain how to start nodes of this type with \texttt{geth} v1.8.11.

\paragraph{Full nodes}
The full nodes are the nodes we describe throughout our work. They store the
whole blockchain (comprising of block headers and bodies), they have full copy
of the most recent states, and they verify and process every transaction. A full
node can be a miner.

When running a full node the user has the possibility to decide whether to use:
\begin{itemize}
    \item the normal synchronization mechanism, i.e. the protocol version
    $62$~\autoref{sec:normal-sync}:
    \begin{center}
        \verb|geth --syncmode full [other-options]|
    \end{center}
    \item the fast synchronization, i.e. the protocol version
    $63$~\autoref{sec:fast-sync}
    \begin{center}
        \verb|geth --syncmode fast [other-options]|
    \end{center}
\end{itemize}
The default option is ``fast'', because it require significantly less time, as
explained in~\autoref{sec:propagation-layer}.

\paragraph{Archive nodes}
Archive nodes are full nodes that store also the state tree for each block,
hence they are the most storage-bound nodes. They are used by block explorers,
like Etherscan\footnote{\url{https://etherscan.io/}} or enterprises that needs
to have at their disposal historical information.

To run an archive node, it is sufficient to overwrite the garbage collection
mode (gcmode) for the state, so that all the intermediate states are stored:
\begin{center}
\verb|geth --syncmode full --gcmode archive [other-options]|
\end{center}

\paragraph{Light nodes}
Light nodes store only the headers of the blocks of the blockchain. They do
neither store the blocks' bodies nor the state. The idea behind light nodes, is
to use the other peers in the network as a distributed hash table (DHT). In
essence, they know the hash of the information and request the desired
information from the peers they know with an on-demand logic.

For example, a light node can retrieve the state of an account at a desired
block by recursively requesting the content of the World State, starting from
its hash which is contained in the block's header (\autoref{fig:world-state}).
We refer to~\cite{bib:light-client} for further details about how and what type
of other information a light node can retrieve.

This type of clients assume that there are full-nodes in the network supporting
the Light Ethereum Subprotocol (LES)~\cite{bib:les-protocol} which have message
types to retrieve data on-demand.

To run a light node it is sufficient to overwrite the default synchronization
method:
\begin{center}
    \verb|geth --syncmode light [other-options]|
\end{center}