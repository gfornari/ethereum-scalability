\subsubsection{Ethereum Wire Protocol}
\label{sec:ethereum-wire-protocol}

The Ethereum Wire Protocol (\emph{eth})~\cite{bib:ethereumwireprotocol}  is an
application level sub-protocol of the \devpp{} Wire Protocol. With the term
sub-protocol, it is intended an extension of the base protocol defining more
messages which are added to the base ones. It is used to spread the information
about the blockchain and for the synchronization.

\paragraph{\devpp{} Wire Protocol}
The base \devpp{} Wire Protocol exploits RLPx Transport Protocol, that is, the
\devpp{} Wire Protocol's messages are sent through the RLPx Protocol in its
payload. \devpp{} Wire Protocol has four message types: \verb+Hello+,
\verb+Disconnect+, \verb+Ping+ and \verb+Pong+.

The \verb+Hello+ (Handshake) message is used upon connection and upon receiving
a message of this kind. This message specifies, among others, the protocol
version, the \textbf{capabilities} and their version, the port on which the
client listens and the node ID. The capabilities are the application level
protocols supported by the sender of the Hello Message. At the moment of
writing, only the \emph{eth} (Ethereum), \emph{les} (Light Ethereum
Subprotocol), \emph{bzz} (Swarm) and \emph{shh} (Whisper) protocols are used.
The \verb+Disconnect+ message notifies the receiver that the sender is going to
disconnect itself. The disconnect message can specify optionally an integer that
encodes a reason. For the complete reason codes, we refer to the \devpp{}
specification~\cite{devp2pwire}. The \verb+Ping+ and \verb+Pong+ messages are
used to check whether the counterpart is still on-line or not.

\paragraph{Ethereum Wire Protocol}
There are several versions of the Ethereum Wire Protocol. Throughout this 
section we will consider only the versions $62$ and $63$ (which are compatible) 
that are currently supported by the Go client implementation of the Ethereum 
protocol\footnote{\url{https://github.com/ethereum/go-ethereum/}}
(v1.8.11).

The first message that must be exchanged between two peers is the
\textbf{Status} message. This kind of message is used to exchange, among others,
the protocol version, the network IDs, the total difficulty of the heaviest
chain known, the hash of the best known block and the genesis block's hash. This
message should be sent only during the handshake phase. If the network id or the
genesis block's hash do not match or the supported \emph{eth} protocol versions
are not compatible, the peers should drop the connection, since they are either
on different chains or are not able to communicate with each other.

\subparagraph{Version 62 - Model Syncing}
To spread the presence of one or more blocks to peers that are not aware of
them, the \textbf{NewBlockHashes} message type is used. Moreover, the
\textbf{Transactions} message type spreads transactions to peers who are not
aware of them. It is specified that in the same session a peer should not send
twice the same transaction to a recipient\footnote{To this extent the geth
implementation (in the file \path{eth/peer.go}) keeps track for each peer of the
set of transactions hash (\texttt{knowTxs}) and the set of block hash
(\texttt{knownBlocks}) known to be known by it.}. The \textbf{GetBlockHeaders}
message type requests to the recipient at most \texttt{maxHeaders} block headers
descendant from the block with a given number or a given hash. The recipient of
the message should respond with a \textbf{BlockHeaders} message, in which it has
the faculty to send a reply with less than \texttt{maxHeaders} headers. Clearly,
if the recipient of a message is not aware of any descendant of the given block,
it sends a valid empty reply. Furthermore, to request and receive the real
content of the blocks, the peers have the \textbf{GetBlockBodies} and
\textbf{Blockbodies} messages. The requester specifies the hashes of the blocks
it wants and the recipient replies with the bodies (i.e. the transactions and
the uncles) of the required block(s). Finally, the \textbf{NewBlock} message
spreads a single new block.

\subparagraph{Version 63 - Fast synchronization}
From version v1.3.1 of
geth\footnote{\url{https://github.com/ethereum/go-ethereum/releases/tag/v1.3.1}}
it is possible to perform a fast synchronization. This synchronization type does
not require that a node performs \emph{all} the computations happened during the
history (i.e. the whole EVM instructions). Indeed the synchronizer downloads
along the blockchain the transaction receipts which encapsulate useful
information about the execution result of the transactions. This allows the
synchronizer to deal only with the verification of the proof-of-work. At least
in geth, this synchronization is possible only by the first synchronization for
security
reasons\footnote{\url{https://github.com/ethereum/go-ethereum/pull/1889}}. After
the synchronizer reaches a \textit{pivot point} (last block $- 1024$) it
retrieves the whole current state and afterwards processes the blockchain
normally. To perform this synchronization the clients have at their disposal the
\textbf{GetReceipts} and \textbf{Receipts} messages, which are the request for
the receipts given the hash and the replies, respectively. Besides these, there
are also the \textbf{GetNodeData} and \textbf{NodeData} message types which
provides the mean to query and send the required version of the state: the first
one message types take as input a variable number of hashes and the second one
responds with the content.