\subsubsection{Ethereum Wire Protocol}
\label{sec:ethereum-wire-protocol}

The Ethereum Wire Protocol (\emph{eth})~\cite{bib:ethereumwireprotocol} is an
application level subprotocol of the RLPx transport protocol. It is used to
spread the information about the blockchain and for the synchronization.

Currently, there are several versions of this protocol. Throughout this section
we will consider only the versions $62$ and $63$ (which are compatible) that are
currently supported by \texttt{geth} (v1.8.11).

The first message that should be exchanged between two peers is the
\texttt{Status} message. This message type is used to exchange information such
as the Ethereum Wire Protocol version, the network ID, the total difficulty of
the heaviest chain known, the hash of the best known block and the genesis
block's hash. This message should be sent only during the handshake phase. If
the network ID or the genesis block's hash does not match or the supported
\emph{eth} protocol versions are not compatible, the peers should drop the
connection since they are either on different chains or are not able to
communicate with each other.

\subparagraph{Version 62 - Model Syncing}
\label{sec:normal-sync}
To spread the presence of one or more blocks to peers that are not aware of
them, the \texttt{NewBlockHashes} message type is used. Moreover, the
\texttt{Transactions} message type spreads transactions to peers who are not
aware of them. It is specified that in the same session a peer should not send
twice the same transaction to a recipient\footnote{To this extent, the
\texttt{geth} implementation (in the file \path{eth/peer.go}) keeps track for
each peer of the set of transactions hash (\texttt{knowTxs}) and the set of
block hash (\texttt{knownBlocks}) known to be known by it.}. The
\texttt{GetBlockHeaders} message type requests to the recipient a specified
amount of block headers descendant from the block with a given number or a given
hash. The recipient of the message should respond with a \texttt{BlockHeaders}
message in which it has the faculty to send a reply with less than the specified
amount of headers. Clearly, if the recipient of a message is not aware of any
descendant of the given block, it sends a valid empty reply. Furthermore, to
request and receive the real content of the blocks, the peers have the
\texttt{GetBlockBodies} and \texttt{Blockbodies} messages. The requester
specifies the hashes of the blocks it wants and the recipient replies with the
bodies (i.e. the transactions and the uncles) of the required blocks. Finally,
the \texttt{NewBlock} message spreads a single new block.

\subparagraph{Version 63 - Fast synchronization}
\label{sec:fast-sync}
From version v1.3.1 of
\texttt{geth}\footnote{\url{https://github.com/ethereum/go-ethereum/releases/tag/v1.3.1}},
it is possible to perform a fast synchronization. This synchronization type does
not require that a node performs \emph{all} the computations happened during the
history (i.e. the whole EVM instructions~\autoref{sec:evm} ). Indeed, the
synchronizer downloads along the blockchain the transaction receipts which
encapsulate useful information about the execution result of the transactions.
This allows the synchronizer to deal only with the verification of the
proof-of-work \autoref{sec:consensus:algorithm}. At least in \texttt{geth}, this
synchronization is possible only by the first synchronization for security
reasons\footnote{\url{https://github.com/ethereum/go-ethereum/pull/1889}}. After
the synchronizer reaches a \emph{pivot point} (i.e. last block minus $1024$), it
retrieves the whole current state from the other peers and subsequently
processes the blockchain normally. To perform this synchronization, the clients
have at their disposal the \texttt{GetReceipts} and \texttt{Receipts} messages,
which are the request for the receipts given the hash and the replies
respectively. Besides these, there are also the \texttt{GetNodeData} and
\texttt{NodeData} message types which provide the mean to query and send the
required version of the state: the first one takes as input a variable number of
hashes and the second one replies with the content.
