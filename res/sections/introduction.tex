\section*{Introduction}

Ethereum is based on the Blockchain technology. This technology is one of the
most active research fields in ICT in the last couple of years, despite the
first appearance of blockchain in the actual form is due to Satoshi Nakamoto's
groundbreaking paper, ``Bitcoin: A peer-to-peer electronic cash system''
(2008)~\cite{bib:bitcoin}. The purpose of this report is to investigate the
architecture and the scalability of Ethereum~\cite{wood2018ethereum}.

The aim of the blockchain technology is to provide a total order of the
transactions in a distributed ledger without relying on a trusted third party,
e.g. a bank~\cite{bib:the-quest}. Not relying on a trusted central authority may
lead to practical issues like transaction repudiation and the infamous
\textbf{double-spending problem}. The former is self-explanatory and can be
solved by digital signatures, while the latter consists in using the same
digital token to pay multiple entities and found its first practical 
decentralized solution with the appearance of 
Bitcoin~\cite{ethereumwp}\footnote{We refer to~\cite{ethereumwp} for a complete
survey about the history and ancestors of Bitcoin.}.

This technology has found, apart from merely financial applications, other
applications such as auctions, supply chain and notary services.

\paragraph{Bitcoin}
Bitcoin is a state transition system, in which there is a transition from a
valid state to another valid state through a valid \emph{transaction}. The state
consists in the balance of the addresses\footnote{The addresses correspond to a
private/public key pair. Each peer of the network can have zero or more
addresses.}. Each node in the network maintains a local copy of the state and
updates its \emph{own} copy of the state in a deterministic way according to the
transactions. Therefore, to have an exact replica in each node, the order of
transactions should be total and agreed by every member of the network. The
mechanism through which this total order is provided and maintained is the
blockchain, which is literally a chain of blocks. Each block of the chain
contains an ordered list of transactions and is connected to the previous block
by inserting the hash of the previous block in its header. Each node of the
network has the faculty to create transactions and have to sign them to show
that it owns the private key corresponding to a given address. The transactions
are spread in the network through gossip protocols. Once a node receives a new
transaction, it verifies that the transaction is well-formed. If it is the case,
the node sends the transactions to the other known peers. Eventually, the
transactions are received by a member of the network who groups some
transactions in a block and tries to find a nonce such that the hash of the
block is smaller than a given value. Since this task is computationally
expensive, if the node finds this value, it adds at the beginning of the
transaction list a transaction in which it assign to a beneficiary address an
amount of newly minted coins, according to the protocol's rule\footnote{This
value was initially $50$ bitcoins. This reward halves every $210000$ blocks.
Currently it is 12.5 bitcoins. Around year $2140$ no coins would be
minted~\cite{bib:masteringbitcoin}.}. In addition to this reward, it receives
also fees from the senders of the included transactions. The members of the
network who try to create new blocks are called miners, because their action
resembles the extraction of precious metals. The miners are incentivized to
create valid blocks, that is, containing valid transactions and the correct
solution to the puzzle, so the other peers of the network can accept \emph{only}
valid blocks. They can verify the correctness of the transactions (e.g. the
balance of the addresses is always positive), because they have a local copy of
the state, and the correctness of the nonce by computing a single hash. It is
worth notice that multiple parties try to create new blocks concurrently,
therefore it is possible that multiple versions of the blockchain co-exist.
Indeed, a mechanism to select the canonical blockchain is needed. In the case of
bitcoin it is simply the longest chain, because it corresponds to the one with
more work invested on it. The co-existence of multiple blockchain can be very
useful in case of a network partition, indeed, once the partition is over, the
peers can agree on the blockchain. The drawback of this system is that there is
no consensus finality~\cite{bib:the-quest}, thus it is necessary to wait a
certain number of blocks (confirmation blocks) to be sure that the transactions
are really confirmed. The number of confirmation blocks in bitcoin is six, which
correspond to approximately one hour.

Although this description of Bitcoin abstracts from various details, it is
sufficient to show how the double spending problem is solved. The idea is to let
each peer of the network know the current state and the transactions that are
already spent. Moreover, after the confirmation time the blockchain can be
considered immutable and tamper-proof, because to rewrite the sufficiently old
transaction history an enormous amount of work should be done. The immutability
is an interesting property that can be used to emit certificates about the
ownership of an asset such as a digital artwork or an intellectual property.

\paragraph{Permissioned vs Permissionless Blockchain}
Apart Bitcoin, a lot of cryptocurrencies, also known as
\emph{altcoins}\footnote{The contraction of "alternative coins".}, came out.
They have different peculiarities, but the general idea is the same as Bitcoin.
In the literature~\cite{bib:the-quest}, it is common to distinct between
permissionless and permissioned blockchains. The former are blockchains in which
everyone can participate in, while the latter requires authentication and is
commonly used by banks or consortium of companies. Prominent examples of
permissionless blockchains are Bitcoin and Ethereum, while a representative
permissioned blockchain is Hyperledger.

\paragraph{Overview}
In the remainder of the paper we will discuss exclusively Ethereum. Ethereum can
be viewed as a generalization of Bitcoin. While in Bitcoin the execution
environment (i.e. the script language) is stateless and is used only to express
conditions to spend the money, e.g. demonstrate the possession of a given
private key, in Ethereum the execution environment (i.e. the EVM) is stateful
and it is Turing Complete. To avoid the misuse of the network resources, each
opcode is associated with an amount of gas, so the termination of the
computations is always guaranteed.

This paper is organized in two main chapters. The first chapter describes the
architecture of Ethereum in a top-down fashion by individuating and describing
the layers in which it can be split. The second chapter analyzes the scalability
of the system, both theoretically, with the analysis of the literature
(\autoref{sec:background}) and the description of the cube of the
scalability~\cite{bib:art-of-scalability} applied to Ethereum
(\autoref{sec:scale-cube}) and practically, by creating a private Ethereum
Network and running some tests (\autoref{sec:tests}).
