\subsection{State}

We can distinguish two different states in the Ethereum system:

\begin{itemize}
  \item the \textbf{World State}, referred simply as the \emph{state}, which is
  a mapping between account addresses and account states (i.e. the blockchain)
  \item the \textbf{Machine State}, the state of an Ethereum Virtual Machine
  (EVM) (see ??)
\end{itemize}

\subsubsection{Accounts}

The accounts are also called the \emph{state objects} and are essential for the
user to interact with the Ethereum blockchain via transactions.

There are two types of accounts:

\begin{itemize}
  \item the \textbf{non-contract account} (referred to as \emph{account})
  \item the \textbf{contract account} (referred to as \emph{contract}), which
  has EVM Code associated with it and is controlled by its contract code
\end{itemize}

A \emph{non-contract account} has no EVM Code associated with it and is
controlled by a private key. This type of account can send a message to another
non-contract account, that is a value transfer, or to a contract account in
order to trigger the execution of code. The state of an account is its balance.

A \emph{contract account} has EVM Code associated with it and is controlled by
it. This type of account cannot send messages on its own, but can only initiate
transactions as a response to a trigger. The state of a contract is its balance
and its contract storage. A contract code is executed by the EVM, can manipulate
its own persistent storage and can send internal transactions (i.e. call
message) to other contracts.


