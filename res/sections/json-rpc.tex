\subsection{External Interaction}

So far we described how Ethereum \emph{internally} works but did not provide
a description of \emph{how} a user or an external application can interact with 
the system, e.g. how transactions are sent by users or how can an application
read the balance of a given address.

To this extent the JSON RPC API has been 
developed.
It allows external actors to call the exposed API methods by performing a
POST requests (e.g. with \texttt{curl}). These requests should specify the
API version, the api method and the parameters encoded as a list and a nonce
that bind a request to a reply.

%\begin{lstlisting}[language=bash]
%	curl -X POST -H "Content-Type: application/json" --data
%	'{"jsonrpc":"2.0","method":"net_peerCount","params":[],"id":1}'
%	http://localhost:8545
%\end{lstlisting} 

In addition to the JSON RPC API, a Javascript API was developed. It is provided
as a Javascript library, 
\texttt{web3}\footnote{\url{https://github.com/ethereum/web3.js}}, that allows
Javascript code to
communicate with a running Ethereum client. It is simply a convenient
Javascript wrapper for the JSON RPC calls~\cite{bib:javascript-api}.

For a complete list of the methods supported by the two APIs we refer to the
respective documentations~\cite{bib:json-rpc, bib:javascript-api}. 

