\subsection{External Interaction}

So far we described how Ethereum \emph{internally} works but we did not provide
a description of \emph{how} a user or an external application can interact with
the system, e.g. how transactions are sent by users or how an application can
read the balance of a given address.

To this extent, the Ethereum community has developed a JSON RPC
API~\cite{bib:json-rpc} compliant with the JSON RPC 2.0
specification~\cite{bib:json2012json}. JSON RPC is stateless and can be used on
top of diverse protocols (e.g. HTTP). It allows external actors to invoke the
exposed API methods by sending JSON encoded requests. These should specify the
API version, the API method, the parameters encoded as a list and a nonce that
binds a request to a reply. For the sake of clarity, we show an example that
calls the method \verb|eth_blockNumber|, which returns the number of blocks:

\begin{lstlisting}[language=bash]
curl -X POST -H "Content-Type: application/json" --data \
'{"jsonrpc":"2.0","method":"eth_blockNumber","params":[],"id":1}' \
http://localhost:8545
\end{lstlisting}

The server replies with a JSON string that contains the result and the same
nonce as the request.

In addition to the JSON RPC API, a JavaScript API was developed. It is provided
as a JavaScript library,
\texttt{web3}\footnote{\url{https://github.com/ethereum/web3.js}}, that allows
JavaScript code to communicate with a running Ethereum client. It is simply a
convenient JavaScript wrapper for the JSON RPC calls~\cite{bib:javascript-api}.
For a complete list of the methods supported by the two APIs, we refer to the
respective documentations~\cite{bib:json-rpc, bib:javascript-api}.

The different Ethereum client implementations provide options to start JSON RPC
server on top of different protocols: for example, go ethereum allows to start
the server on top of HTTP, WebSocket and IPC Socket, i.e. shared memory.
Moreover, this client gives the possibility to users to start or to attach to a
given running instance a JavaScript Runtime Environment REPL
console~\cite{bib:js-console}, which can execute Javascript programs and access
the JavaScript API methods.
