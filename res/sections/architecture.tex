\section{Architecture}

Describing the Ethereum architecture, first we want to underly the component it
is comprised of, and subsequently we want to outline the relations between them
from a functional point of view.

Ethereum is a peer-to-peer architecture, a decentralized architecture in which
the nodes are logically equivalent and function as a \emph{servent} (i.e. a that
node acts as a client and a sever at the same time). In this type of
architecture, the nodes are formed by processes and the links represent the
possible communication channels, that is, an \emph{overlay network}
\cite{van2017distributed} which is discussed in \autoref{sec:overlay-network}.
The peer-to-peer architectures support the \emph{horizontal distribution} for
which each node operate on its own share of the complete data set, thus
balancing the load. Ethereum implement it differently, infact each node read and
write its own copy of the complete data set, but rather to achive consesus of
the state instead of balancing the load.


