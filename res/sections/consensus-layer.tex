\subsection{Consensus layer}
\label{sec:consensus}

The ultimate aim of the blockchain technology is to provide a \textbf{total
order to transactions} in a distributed ledger~\cite{bib:the-quest} without
relying on a trusted third party. This permits to solve the double spending
problem~\cite{bib:bitcoin}. Moreover, in Ethereum the order of transactions can
also affect the execution of smart contracts by altering the content of their
storage.

In order to describe how the nodes reaches the consensus, we briefly describe
the algorithm to agree on the transaction order
(\autoref{sec:consensus:algorithm}) and thereafter we describe the common state
transition procedure, that describe how to move to a new valid state given a
transaction (\autoref{sec:tx-execution}), that can be either a contract creation
(\autoref{sec:contract-creation}) or a message call
(\autoref{sec:message-call}).


\subsubsection{Consensus Algorithm}
\label{sec:consensus:algorithm}

Finding an agreement on the order of transaction (i.e. the actual blockchain)
and the world status is crucial, thus multiple consensus algorithms were
proposed~\cite{bib:the-quest}. Ethereum follows an idea very close to the
consensus algorithm of Bitcoin, which is also known in the literature as
\textbf{Nakamoto consensus}~\cite{bib:bitcoin-ng}.

The basic idea of this algorithm consists in:
% enumerate* means in-line enumeration
\begin{enumerate*}[label=(\arabic*)]
    \item accepting only valid blocks with regards to some validation criterion,
    \item create new valid blocks by using a proof-of-work algorithm,
    \item relying on a selection rule to choose between two different
    valid forks, depending on the amount of work performed in each fork.
\end{enumerate*}

\paragraph{Validation}
The \emph{validation criterion} used to determine whether a block is valid or
not consists in:
\begin{enumerate}
    \item checking that the blocks and transactions are well-formed,
    \item re-performing all the transactions,
    \item re-executing \emph{all} the EVM computations to verify whether the
    transaction receipts and the state root contained in the propagated block
    (\autoref{fig:world-state}) are valid, i.e. corresponds to the values
    computed locally,
    \item checking that the nonce in the block is valid.
\end{enumerate}

\paragraph{Proof of Work}
Ethereum uses an improved version of the
Dagger-Hashimoto algorithm~\cite{bib:dagger-hashimoto}, known as
Ethash~\cite[Appendix J]{wood2018ethereum} as PoW algorithm.
The rationale to use of this memory intensive algorithm is its
ASIC-resistance. ASIC are specialized hardware used massively in the Bitcoin
ecosystem. This kind of Hardware is a risk for centralization,
because to begin mining new blocks and maintaining the infrastructure a big
initial investment is needed and only few entities and definitely not private
parties can afford this cost.

Essentially to create a valid block the miner should find a mixHash and
a nonce (\autoref{fig:world-state}) for the block.
The PoW algorithm takes as input the block header without nonce,
the candidate nonce and a big dataset, known as \textbf{DAG} and
returns the mixHash and a number $n$. The puzzle is resolved if
$n$ is smaller than $2^{256}$ divided by the difficulty of the block. Clearly,
the higher the difficulty the higher the number of nonce to try to find the
right values. The DAG can be precalcolated and is fixed for each epoch, i.e.
$30000$ blocks, that corresponds roughly to $100$ or $141$ hours\footnote{
These values are obtained by considering a new valid block each $12$
seconds~\cite{bib:} and $17$ seconds~\cite{bib:solidity}, respectively.}. To
verify that the mixHash and the nonce are valid only a cache for the DAG is
needed. Actually, the cache is required also to generate the DAG itself.
At each epoch the DAG and the cache change and their size increase of
$80$ mebibytes and $128$ kibibytes, respectively.
We refer to the yellow paper~\cite[Appendix J]{wood2018ethereum} to get more
details on how these data structures are computed.



\paragraph{Selection Rule}
The \emph{selection rule} is required to avoid the infamous double spending
problem. Indeed, in Bitcoin (and as well in Ethereum) the assumption is that the
majority of computing power belongs to good players who will follow the rules.
Therefore, in order to prevent bad agents to rewrite the transactions history
with a high probability~\cite{bib:bitcoin}, the issuer of new blocks should
prove that she invested resources in its creation by solving a computational
heavy task. This mechanism is known as \textbf{Proof-of-Work} (PoW).



In Bitcoin
the selection rule consists in accepting the longest chain that corresponds
roughly to the one with more work invested on it.
The Ethereum
community \emph{claims} that Ethereum implements a simplified version of the
Greedy Heaviest-Observed Sub-Tree (GHOST) selection
rule~\cite{wood2018ethereum}:
briefly, the stale blocks contributes to the difficulty of a fork.
The aim is to allow an increase in the transaction throughput (by decreasing
the block issue interval) while preserving the same security guarantees of the
original bitcoin consensus protocol.
As noted in~\cite{bib:securityAndScalabilityPoW} Ethereum \emph{does not}
implement a simplified version of the GHOST selection rule.
Indeed,
currently the Ethereum's rule consists in choosing the fork with the
highest accumulated difficulty~\cite{wood2018ethereum}.
Each block in the chain has an
associated difficulty that determines how much effort is needed to mine a new
block. This parameter depends \emph{solely} on the difficulty of the previous
block and the time that elapsed between the previous block's timestamp and the
new block's timestamp, corrected with some bounds to avoid sudden decreases or
increases in this value. The claims to use the GHOST rule are motivated by the
fact that the headers of stale blocks (up to six blocks before, the
\emph{ommers}) can be included in the blockchain and rewarded, but they do
neither contribute to the difficulty of the blocks nor are verified to be
valid~\cite{bib:securityAndScalabilityPoW}. Thus, at the state of the art
this rule resembles the bitcoin one.




%In the case of bitcoin this task consists in finding a number (\emph{nonce})
%such that the hash of the block along with the nonce itself begins with a
%number of leading zeros. The difficulty of the task increase exponentially with
%the number of leading zeros. Since the output of the hash function is not
%predictable, the creator of a new block should try each possible nonce in a
%brute-force manner. Once a valid nonce is found the block is spread and the
%receiver should only perform a single hash to check that the result is valid.

