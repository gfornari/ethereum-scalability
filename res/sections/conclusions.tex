\section{Conclusions}
\label{sec:conclusions}
The Blockchain technology provides a way to find a total order of transactions
in a distributed system without relying on a trusted third party. This allows
to have an exact copy of the state in each node of the network, because each
node starts from the same state and changes the state according to the
transactions, whose execution is deterministic. The main advantage of this
technology with respect to traditional transition systems is its decentralized
nature.

In this report we took into consideration Ethereum, a representative of the
permissionless blockchain technology. This system is particularly interesting
because it supports a general-purpose execution environment, the EVM.

In the first part of this work we proposed a decomposition of the Ethereum's
architecture in logical layers. The five stacked layers abstract from the
implementation details and offer a conceptual organization useful to compare
different blockchain proposals, although enabling flexibility as the case of the
EVM which is a vertical layer (see \autoref{sec:architecture}. Moreover, this
decomposition allows component-oriented development, which helps reasoning on the
system overall.

The focus of the second part is the analysis of the scalability of Ethereum. To
do so, we analyzed the existing literature (\autoref{sec:background}) and
confirmed empirically that the current version of Ethereum reaches a maximal
transaction throughput of approximately 15 transactions per second, even if the
transactions do not require computations. This value is not influenced by the
number of miners, but rather by the block size and the block interval.
Furthermore, we analyzed the motivation of this reduced transaction throughput
with the aid of the cube of scalability~\cite{bib:art-of-scalability}. We argue
that the current version of Ethereum is not developed in any direction of the
cube by analyzing the current specification. Thereafter, we categorize the
scalability improvement proposals based on the axes they will affect. From this
analysis, it is clear that the most efforts of the community are concentrated on
the z-axis of the cube (see \autoref{sec:z-axis}) with proposals like Plasma and
sharding. Sharding is a scaling strategy already seen in the database systems,
that is, systems which have to manage persistent data. Maybe, this affinity
between Ethereum and the database systems leads the z-axis of the Scale Cube.

The result of our research maps directly to the scalability
trilemma~\cite{bib:sharding-faq}, which states that we cannot have at the same
time scalability, decentralization and security. Assuming full security, we need
to find a trade-off between scalability and decentralization. The current
implementation of the Ethereum system provides security and decentralization and
gives up scalability.

Proposals like Proof-of-Stake (see~\autoref{sec:pos}) increase the transaction
throughput but, since the performance improvement does not depend on the number
of nodes in the system, we cannot view PoS as an improvement from the
scalability point of view. However, PoS increments the decentralization because
allows nodes with commodity and not specialized hardware to be validators.

Sharding and Plasma provide data partition allowing parallel processing of
transactions in different shards or child chains, thus increasing the
scalability of the system with respect to the number of shards or child chains.
However, the decentralization decreases in both proposals, since in Plasma there
are not nodes having the blocks of the entire network, and in Sharding due to
the introduction of super nodes (see~\autoref{sec:sharding-ethereum}).

A technology like blockchain cannot give up security, which represents from
the very beginning one of the main desired properties in these types of
systems. Another interesting property promoted by the blockchain is
decentralization. This property is more difficult to understand in terms of
advantages which provides and costs to be borne in order to sustain it. Albeit
the expectation around this technology was and is to adapt and implement any
application, the scalability was a clear issue but postponed. Now, the challenge
is to increase the performance of the system overall without loosing security,
hence finding the best trade-off between decentralization and scalability,
possibly creating different solutions for different applications.
