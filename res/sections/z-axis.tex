\subsubsection{Z-Axis: Horizontal Data Partitioning}

The direction taken by scaling on the y-axis is to segment based on the service,
i.e. based on \emph{dissimilar} things. Scaling on the z-axis means segment on
\emph{similar} things, thus making the segmentation biased by the data or the
actions that are unique to the sender or the receiver of the request. In
particular, to enhance significantly the system a z-axis split should partition
both transactions and the data necessary to perform the transactions. 
An example of such scaling could be, in a client-server architecture, the 
geographic distribution of the servers based on the clients requests, such that 
a request performed in Europe is undertaken by a server located in Europe 
instead of one located on the other side of the globe. One similar example of
successful z-axis split is sharding.

\paragraph{Sharding} 
Sharding is commonly used in distributed databases to scale out by splitting
the data of a database in several servers. One emblematic example is
MongoDB\footnote{\url{https://www.mongodb.com/}}~\cite{bib:mongodb}. This kind
of NoSQL database splits the documents into the different \emph{shards}
according to a selected \emph{shard key}. Each shard is accountable to store 
the documents in a partition of the key space. To access the data 
transparently, that is without knowing where the data are really stored an
entity, called \texttt{mongos} is introduced. Its aim is to redirect the
requests to the right shards, collect the responses and return them to the
requester. Moreover a balancer is introduced. It is accountable to balance the 
amount of data in each server while minimizing the transfer of data among the 
shards. When a shards is too full it may create new shards, if additional
servers are available, or it may simply rearrange the partition. 
The big advantage of MongoDB's approach is that the splitting of data and
migration of data are done automatically without human intervention.






\paragraph{Ethereum} We argue that currently Ethereum is not developed in the
z-axis, because each node of the network should have information about the whole
blockchain and the status of all accounts in the network 
(\autoref{sec:world-state}) to process the transactions. For this reason 
Ethereum cannot be more efficient than a single machine, as already pointed out 
by Vitalik Buterin in the muave paper~\cite{bib:mauve}.

Furthermore, a z-axis split would be not trivial, as the following example 
clarifies. Let's consider the process of a transaction that requires some 
computations on the EVM. To complete this action we need several information, 
such as the balance of the sender of the transaction and the code of the
invoked contract. Moreover, the invoked contract may in turn call other 
contracts and so on. These may in addition modify the balance of other accounts
as a side effect, e.g. through the \texttt{SELFDESTRUCT} opcode. Thus, to 
process a transaction we need the account states of the sender and the called
contracts and all account state that are affected by the computations.

\paragraph{Proposals} To tackle a z-axis split the Ethereum 
foundation~\cite{bib:mauve} and the scientific 
community~\cite{bib:scaling-croman} has proposed sharding as an effective 
measure to augment the transaction throughput.

The Ethereum sharding proposal consists in splitting the world state and
transaction history into different partitions, called shards. 
Each shard is a different universe in which the transactions affect only the 
accounts in the same shard. This allows to augment dramatically the transaction
throughput, by parallelizing the processing of transactions.
Obviously, it is desirable to allow the communication between different shards,
therefore \emph{cross-sharding} communication mechanisms were proposed.

Currently, the prysmaticlabs company is modifying the go ethereum implementation
to implement both PoS and sharding.




