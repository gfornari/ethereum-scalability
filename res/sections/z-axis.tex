\subsubsection{Z-Axis: Horizontal Data Partitioning}

The direction taken by scaling on the y-axis is to segment based on the service,
i.e. based on \emph{dissimilar} things. Scaling on the z-axis means segment on
\emph{similar} things, thus making the segmentation biased by the data or the
actions that are unique to the sender or the receiver of the request. In
particular, to enhance significantly the system a z-axis split should parition
both transactions and the data necessary to perform the transactions. An example
of such scaling could be, in a client-server architecture, the geographic
distribution of the servers based on the clients requests, such that a request
performed in Europe is undertaken by a server located in Europe instead of one
located on the other side of the globe.


\paragraph{Ethereum} We argue that currently Ethereum is not developed in this
axis, because each node of the network should have information about the whole
blockchain and the status of all accounts in the network (\autoref{sec:state})
to process the transactions. Indeed, to process a transaction that requires the
EVM, at least the balance of the creator of the transaction and the code of the
invoked contract are needed.
Moreover, the invoked contract may in turn call other contracts and so on.
Thus, to process a transaction at least the account state of the sender and the
called contracts should be known. In addition the execution of the contracts may
increase the balance of other accounts as a side effects, e.g. through the
\texttt{SELFDESTRUCT} opcode.
