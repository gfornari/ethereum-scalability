\subsubsection{Z-Axis: Horizontal Data Partitioning}

The direction taken by scaling on the y-axis is to segment based on the service,
i.e. based on \emph{dissimilar} things. Scaling on the z-axis means segment on
\emph{similar} things, thus making the segmentation biased by the data or the
actions that are unique to the sender or the receiver of the request. In
particular, to enhance significantly the system a z-axis split should parition
both transactions and the data necessary to perform the transactions. An example
of such scaling could be, in a client-server architecture, the geographic
distribution of the servers based on the clients requests, such that a request
performed in Europe is undertaken by a server located in Europe instead of one
located on the other side of the globe.


\paragraph{Ethereum} We argue that currently Ethereum is not developed in this
axis, because each node of the network should have information about the whole
blockchain and the status of all accounts in the network 
(\autoref{sec:world-state}) to process the transactions. For this reason 
Ethereum cannot be more efficient than a single machine, as already pointed out 
by Vitalik Buterin in the muave paper~\cite{bib:mauve}.

However, a z-axis split would be not trivial, as the following example 
clarifies. Let's consider the process of a transaction that requires some 
computations on the EVM. To complete this action we need several information, 
such as the balance of the sender of the transaction and the code of the
invoked contract. Moreover, the invoked contract may in turn call other 
contracts and so on. These may in addition modify the balance of other accounts
as a side effect, e.g. through the \texttt{SELFDESTRUCT} opcode. Thus, to 
process a transaction we need the account states of the sender and the called
contracts and all account state that are affected by the computations.

\paragraph{Proposals} To tackle a z-axis split the Ethereum community has 
proposed to implement sharding~\cite{bib:mauve}
