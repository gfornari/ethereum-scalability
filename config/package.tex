\documentclass[12pt,                    % corpo del font principale
               a4paper,                 % carta A4
               english
               ]{scrartcl}


% codifica di input; anche [latin1] va bene
% NOTA BENE! va accordata con le preferenze dell'editor
\usepackage[utf8]{inputenc}

%**************************************************************
% Importazione package
%**************************************************************

%\usepackage{amsmath,amssymb,amsthm}    % matematica


% per scrivere in italiano e in inglese;
% l'ultima lingua risulta predefinita
\usepackage[english]{babel}


\usepackage{caption}                    % didascalie

% gestisce automaticamente i caratteri (")
\usepackage{csquotes}

\usepackage{epigraph}                   % per epigrafi

\usepackage{eurosym}                    % simbolo dell'euro

% codifica dei font:
% NOTA BENE! richiede una distribuzione *completa* di LaTeX
\usepackage[T1]{fontenc}

% rientra il primo paragrafo di ogni sezione
\usepackage{indentfirst}

\usepackage{graphicx}                   % immagini

\usepackage{subcaption}                 % subfigure

\usepackage{hyperref}                   % collegamenti ipertestuali


\usepackage{listings}                   % codici

\usepackage{microtype}                  % microtipografia

\usepackage{mparhack,relsize}  % finezze tipografiche


%~ \usepackage[font=small]{quoting}        % citazioni

\usepackage[dvipsnames]{xcolor}         % colori

\usepackage{tabularx}                   % tabelle di larghezza prefissata


% permette di inserire le immagini/tabelle esattamente dove viene usato il
% comando \begin{figure}[H] ... \end{figure}
% evitando che venga spostato in automatico
\usepackage{float}

%~ % per i todo
\usepackage{todonotes}

\usepackage[inline]{enumitem}

\usepackage{textgreek}
\usepackage{algorithm}
\usepackage{algorithmicx}
\usepackage{algpseudocode}
